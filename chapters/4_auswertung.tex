\chapter{Evaluation}
%
\section{Boost Converter}
    The characteristic curve of a BC gives the dependency of its output voltage with respect to the switching duty cycle.
    A plot of the data points captured shows \cref{fig:characteristic-curve-of-the-boost-converter}.\par
    %
    \begin{figure}[h]
        \centering
        % \includegraphics[width=\linewidth]{messdaten/Characteristic_Curve_of_the_Boost_Converter.jpg}
        \includesvg[inkscapelatex=false, width=.8\textwidth]{scidavis/Characteristic_Curve_of_the_Boost_Converter_mod}
        \caption[Characteristic curve of the BC]{Characteristic curve of the BC.}
        \label{fig:characteristic-curve-of-the-boost-converter}
    \end{figure}
    %
    \begin{equation}
        U_{+}(dc)=\frac{\SI{1.51}{V}}{\SI{100}{\percent}} \cdot dc + \SI{9.07}{V}
    \end{equation}\par
    %
    In comparison with the simulated output voltages at the higher end of the duty cycle deviations of \( \approx \SI{25}{V} \) can be observed.
    On the lower end however, deviations decrease.
    %
\section{Avalanche Pulse Generator}
    The signal in \cref{subfig:osci:avalanche_pulse_signal} is the first that appears after increasing the voltage to the minimum
    voltage $ U_{min}=\SI{70}{V} $. As discussed in \cref{sec:A3_charge-discharge-time} the time period at which the pulses
    occur and, thus, the repetition frequency is a function of the BC's output voltage \( U_+ \). Examining the
    actual behavior yields \cref{fig:repetition-frequency} (see \cref{subtab:3-1_duty_vs_voltage} for reference).\par
    %
    \begin{figure}[H]
        \centering
        % \includegraphics[width=1\linewidth]{messdaten/Repetition Frequency.jpg}
        \includesvg[inkscapelatex=false, width=.8\textwidth]{scidavis/repetition_frequency_mod}
        \caption[Examination repetition frequency over supply voltage]{Examination of the repetition frequency as a function of the supply voltage.}
        \label{fig:repetition-frequency}
    \end{figure}
    %
    The theoretical value for the repetition frequency at $ U=\SI{75}{V} $ with $ f_{Rep}=\SI{225.7}{kHz} $ (see \cref{sec:A3_charge-discharge-time})
    is larger by a factor. In the diagram in \cref{fig:repetition-frequency} the value is around
    $ \SI{90}{kHz} $. So there must be some deviations.\par\medskip
    Next the characteristics of the pulse at a voltage of $ U=\SI{75}{V} $ are examined. \Cref{subfig:osci:pulsuntersuchung}
    gives the following:\par
    %
    \begin{align}
        \text{Amplitude:}\qquad \hat{U}&=\SI{7.12}{V}\\
        \text{Rise time:}\qquad t_r&=\SI{2.33}{ns}\\
        \text{Fall time:}\qquad t_f&=\SI{7.08}{ns}\\
        \text{Pulse width:}\qquad t_w&=\SI{6.26}{ns}
    \end{align}\par
    %
    In comparison to a wave form captured with an oscilloscope with a Bandwidth of \SI{500}{MHz}, it can be seen that the secondary pulsations get suppressed
    quite significantly. Moreover, the signal features higher rise and fall time as well as having a lower amplitude.
    This is an expected behavior as discussed in \cref{sec:A5_oscopes_suitability}.\par
    %
    \begin{figure}[H]
        \centering
        \includegraphics[width=.4\textwidth]{messdaten/500mhz_waveform.jpg}
        \caption[Pulse captured with a \SI{500}{MHz} oscilloscope]{A single pulse output by the APG and captured with an oscilloscope with \SI{500}{MHz}.}
        \label{fig:500MHz_waveform}
    \end{figure}
    %
\section{Signal Propagation}
%
    \subsection{Propagation Time}
        The propagation speed of a signal through a transmission line with known length is given by\par
        %
        \begin{equation}
            c_p=\frac{l}{t}
            \label{eq:propagation_speed}
        \end{equation}\par
        %
        The measured propagation times and the calculated propagation speeds are summarized in \cref{tab:delayed_signal}\par
        %
        \begin{table}[H]
            \caption{Delayed signal}
            \centering
            \begin{tabular}{@{}llll@{}}
                \toprule                                    
                Cable no.   & $ l \pm \Delta l $ \big/ \SI{}{m} & $ t \pm \Delta t $ \big/ \SI{}{ns}    & $ c_p \pm \Delta c_p $ \big/ \( \SI{}{\frac{m}{s}} \) \\ \midrule
                1           & $ 4.95 \pm 0.05 $                 & $ 22.6 \pm 1 $                        & $ (2.19\pm 0.12)\cdot 10^8 $ \\
                2           & $ 10.04 \pm 0.05 $                & $ 44.8 \pm 1 $                        & $ (2.24\pm 0.06)\cdot 10^8  $ \\
                3           & $ 0.77 \pm 0.01 $                 & $ 4.6 \pm 1 $                         & $ (1.67\pm 0.39)\cdot 10^8 $ \\ \bottomrule
            \end{tabular}
            \label{tab:delayed_signal}
        \end{table}
        %
        The error in propagation speed can be determined as follows:\par
        %
        \begin{align}
            \Delta c_p&=\left|\frac{\partial c_p}{\partial l}\right|\cdot \Delta l + \left|\frac{\partial c_p}{\partial t}\right|\cdot \Delta t \nonumber \\
            &=\frac{1}{t}\cdot \Delta l + \frac{l}{t^2} \cdot \Delta t
            \label{eq:cp_dev}
        \end{align}\par
        %
        For cable no. 1 in \cref{tab:delayed_signal} e.g.:\par
        %
        \begin{align}
            \Delta c_{p,1}&=\frac{1}{\SI{22.6}{\cdot 10^{-9}s}}\cdot \SI{0.05}{m} + \frac{\SI{4.95}{m}}{(\SI{22.6}{\cdot 10^{-9}s})^2} \cdot \SI{1}{\cdot 10^{-9}s} \nonumber \\
            &=\SI{0.02}{\cdot 10^8\frac{m}{s}}+\SI{0.10}{\cdot 10 ^8\frac{m}{s}} \nonumber \\
            &=\SI{0.12}{\cdot 10^8\frac{m}{s}}
        \end{align}\par
        %
    \subsection{Cable Characteristics}
        %
        \cref{subfig:osci:3-3-2_open_reflectionTime} shows the reflected pulse signal and \cref{subfig:osci:3-3-2_shorted} the shorted and reflected
        pulse signal. The speed at which an electromagnetic signal propagates through a medium can be calculated by\par
        \begin{equation}
            c_p = \left(\sqrt{\varepsilon_0\varepsilon_r}\right)^{-1} \left(\sqrt{\mu_0\mu_r}\right)^{-1}
        \end{equation}\par
        where \( \left(\sqrt{\varepsilon_0\mu_0}\right)^{-1} \) is the speed of light. Sent through a medium other than
        vacuum the electric and magnetic properties of the medium reduce the propagation speed by a factor of \( \left(\sqrt{\varepsilon_r\mu_r}\right)^{-1} \) \cite{Halliday.2005}.
        A metric to quantify the relative speed of propagation is the velocity factor:\par
        \begin{equation}
            VF = \frac{c_p}{c_0}
        \end{equation}\par
        At the minimum amplitude of the reflected pulse, the resistance \( R_{Term} \) equals \( Z_0 \).\par
        The determined and calculated characteristic values are summarized in the following table:\par
        %
        \begin{table}[h]
            \caption{Reflected signal}
            \begin{adjustbox}{width=\textwidth,keepaspectratio}
                \begin{tabular}{@{}llllllll@{}}
                    \toprule
                    Cable no.   & $ l $ \big/ \SI{}{m}  & $ \tau_0 $ \big/ \SI{}{ns}    & $ \tau_s $ \big/ \SI{}{ns}    & $ c_p $ \big/ $ \SI{}{\frac{m}{s}} $  & $ VF $ \big/ \%   & $ Z_0 $ \big/ $\Omega$    & $\varepsilon_r$ \\ \midrule
                    1           & $ 4.95 \pm 0.05 $     & $ 44.8 \pm 1 $                & $ 44.8 \pm 1 $                & $ (2.21\pm 0.07)\cdot 10^8 $          & $ 73.7 \pm 2.3 $  & 80.2                      & $ 1.84 \pm 0.11 $ \\
                    2           & $ 10.04 \pm 0.05 $    & $ 89.6 \pm 1 $                & $ 90.4 \pm 1 $                & $ (2.24\pm 0.04)\cdot 10^8 $          & $ 74.7 \pm 1.3 $  & 56.0                      & $ 1.79 \pm 0.06 $ \\
                    3           & $ 0.77 \pm 0.01 $     & $ 8.4 \pm 1 $                 & $ 10.8 \pm 1 $                & $ (1.83\pm 0.34)\cdot 10^8 $          & $ 61.0 \pm 11.3 $ & 56.4                      & $ 2.69 \pm 1.00 $ \\ \bottomrule
                \end{tabular}
            \end{adjustbox}
            \label{tab:reflected_signal}
        \end{table}
        %
        The deviation of $ c_p $ can be determined similar to \cref{eq:cp_dev}, but with a factor of 2. The delta in $ VF $ as follows:\par
        %
        \begin{align}
            \Delta VF&=\left|\frac{\partial VF}{\partial c_p}\right|\cdot \Delta c_p \nonumber \\
            &=\frac{1}{c_0}\cdot \Delta c_p
        \end{align}\par
        %
        For cable no. 1 in \cref{tab:reflected_signal} e.g.:\par
        %
        \begin{equation}
            \Delta VF_1=\frac{1}{\SI{3}{\cdot 10^8 \frac{m}{s}}}\cdot \SI{0.07}{\frac{m}{s}}=2.3\%
        \end{equation}\par
        %
        And for $ \varepsilon_r $:\par
        %
        \begin{equation}
            \Delta \varepsilon_r = \left|\frac{\partial \varepsilon_R}{\partial VF}\right|\cdot \Delta VF = \frac{2}{VF^3} \cdot \Delta VF
        \end{equation}\par
        %
        So for the first cable in \cref{tab:reflected_signal}:
        %
        \begin{equation}
            \Delta \varepsilon_{r,1}=\frac{2}{0.737^3}\cdot 0.023=0.11
        \end{equation}\par
        %
        If the values are compared with the data sheet from the manufacturer, there can be seen some similarities and differences.
        The determined velocity factor of cable no. 1 and 2 is very close to the stated one (69.5\%). The third cable is slightly
        different.\par
        On the other hand, the impedance is given as $ \SI{53}{\Omega} $, with the second and third cable having a similar value,
        whereas the first cable differs somewhat.
        %
    \subsection{Time Domain Reflectometry}
        For determining the unknown length and the position of the fault of the cable, the equation for the propagation speed has
        to be transformed to the length as\par
        %
        \begin{equation}
            l=\frac{1}{2}\cdot c_p \cdot \tau
            \label{eq:length}
        \end{equation}\par
        %
        $\tau_{total}$ has been measured as \SI{276}{ns} and $\tau_{fault}$ as \SI{256}{ns}.
        For the propagation speed the mean value $ \bar{c_p} = \SI{2.09 \cdot 10^8}{\frac{m}{s}} $ is used. The values inserted in
        \cref{eq:length} results:\par
        %
        \begin{align*}
            l_{total}&=\frac{1}{2}\cdot\SI{2.09}{\cdot 10^8\frac{m}{s}}\cdot \SI{276}{\cdot 10^{-9} s}=\SI{28.84}{m}\\
            l_{fault}&=\frac{1}{2}\cdot\SI{2.09}{\cdot 10^8\frac{m}{s}}\cdot \SI{256}{\cdot 10^{-9} s}=\SI{26.75}{m}
        \end{align*}\par
        %
        The deviation is determined with the following equation:\par
        %
        \begin{align}
            \Delta l&=\left|\frac{\partial l}{\partial c_p}\right|\cdot \Delta c_p + \left|\frac{\partial l}{\partial \tau}\right|\cdot \Delta \tau\\
            &=\frac{1}{2}\cdot \tau \cdot \Delta c_p + \frac{1}{2}\cdot c_p \cdot \Delta \tau
        \end{align}\par
        %
        For the total length and the position of the defect:\par
        \begin{align*}
            \Delta l_{total}&=\frac{1}{2} \cdot\SI{276}{\cdot 10^{-9}s}\cdot \SI{0.15}{\cdot 10^8\frac{m}{s}}  + \frac{1}{2}\cdot \SI{2.09}{\cdot 10^8\frac{m}{s}} \cdot\SI{1}{\cdot 10^{-9}s}\\
            &=\SI{2.07}{m}+\SI{0.10}{m}\\
            &=\SI{2.17}{m}\\
            \\
            \Delta l_{fault}&=\SI{2.02}{m}
        \end{align*}\par
        %
        Summarized the total length is\par
        %
        \begin{equation}
            l_{total}=\SI{28.8}{m} \pm \SI{2.2}{m}
        \end{equation}\par
        %
        and the position of the fault is at\par
        %
        \begin{equation}
            l_{fault}=\SI{26.8}{m} \pm \SI{2.0}{m}
        \end{equation}