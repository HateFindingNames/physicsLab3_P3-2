\chapter{Conclusion}
%
With respect to the simulation of the BC, deviations got more severe the higher the duty cycle gets. As of now we can not
pinpoint the origin of the error, but it is assumed, that the roots spread over the nature of the circuit and our measuring equipment.
Taken the error solely depends on the error introduced by the multimeter, a delta relative to the measured voltage plus a
fixed offset is expected. In fact the deviation at higher duty cycles excel the expected deviation by one order of magnitude.
Though we compared our data points with the output of the simulated circuit, the values and tolerances of the real-world
circuit remain uncertain.

TDR in general is an interesting technique to determine the position of a defect. Without further research done it
is to be questioned how well it can be adopted to other use cases. I.e. is it practical to find a leak in a built-in water
pipe replacing the coaxial cable with the water pipe and sending a sonic pulse instead an electromagnetic one.

While the working principle of the APG is rather simple, it was hard to find any explanation about the physics behind it.
Hence, the \textit{abusive} operation mode of the transistor no ordinary data sheet gives characteristics about the break
through behavior. What we could deduce is that the transistor does not suddenly regenerate as soon as \( U_{CE} \)
falls below the threshold voltage. Instead, there has to be some kind of hysteresis. This situation made it difficult to
explain the difference between the calculated repetition frequency and the observed one. Unfortunately time was too short
to dive deeper into the problem.

The most intriguing aspect of the experiment was to realize how many parameters have to be taken into account using an
oscilloscope. Sure, participating on a workshop about this very instrument would be as much fun as it would be useful.
\medskip
In summary, one can say that the experiment was successful. The boost converter shows similar values for lower voltages
as the simulation had calculated.
The measured value for the repetition frequency deviates from the calculated value by
quite a bit. In general, all graphs recorded at the oscilloscope corresponded to the expected curve. The propagation
times in the cable could be determined well. The cable characteristics are close to the values given in the data sheet.
The Time Domain Reflectometry experiment clearly demonstrated how cable damage can be detected and repaired. The evaluation
of the test result is difficult, since the actual length of the cable and the position of the fault are not given as a
reference in the experiment instructions.