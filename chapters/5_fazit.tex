\chapter{Conclusion}
%
With respect to the simulation of the BC, deviations got more severe the higher the duty cycle gets. As of now we can not
pinpoint the origin of the error but it is assumed, that the roots spread over the nature of the circuit and our measuring equipment.
Taken the error solely depends on the error introduced by the multimeter, a delta relative to the measured voltage plus a
fixed offset is expected. In fact the deviation at higher duty cycles excel the expected deviation by one order of magnitude.
Though we compared our data points with the output of the simulated circuit, the values and tolerances of the used circuit
remain uncertain.

In summary, one can say that the experiment was successful. The boost converter shows similar values for lower voltages
as the simulation had calculated.
The measured value for the repetition frequency deviates from the calculated value by
quite a bit. In general, all graphs recorded at the oscilloscope corresponded to the expected curve. The propagation
times in the cable could be determined well. The cable characteristics are close to the values given in the data sheet.
The Time Domain Reflectometry experiment clearly demonstrated how cable damage can be detected and repaired. The evaluation
of the test result is difficult, since the actual length of the cable and the position of the fault are not given as a
reference in the experiment instructions.